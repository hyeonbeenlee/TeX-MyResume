\section*{지원 동기}
꼼데가르송 한남 FSS 주니어 스태프 지원자 이현빈입니다. 대학원 공학석사학위 취득하였으나, 삶에서 패션문화를 더 가까이 두며 살아가고자 지원하게 되었습니다.

공학계열 학사 및 석사 과정을 수학하며, 많은 전문지식 습득과 프로젝트 수행 및 논문출판에 따른 성취감을 느꼈습니다. 그랬던 저를 지금의 결정에 저를 도달하게 한 것은 과도한 업무량도 잦은 야근과 출장이 아닙니다. 제가 사랑하는 문화를 함께 향유할 사람이 없었다는 점입니다.

공학에 몸담으며 가장 힘겨웠던 점은 그런 주변 환경이었습니다. 함께 직접적으로 패션을 주제로 이야기하지 않아도, 서로가 서로를 알아채고 이해할 수 있는 사람들이 곁에 있어주길 바랬습니다. 나름대로 노력해보았지만, 업무에 치여 그럴 기회나 시간이 그리 많지 않았고 점차 웃는 날들이 줄어가기만 했습니다.

일련의 시간과 생각들을 거쳐 졸업한 뒤로는 주변을 제가 좋아하는 것들로 채워나가기로 결정했습니다. 내가 좋아하는 것이니 재미있고 멋있기만 할 것이라는 헛된 기대나 환상을 품고 들어서지는 않았습니다. 주변에 패션업계나 세일즈 직무에 종사하는 친구들이 다들 각자 근무 스케줄이나 고객응대, 매장 내 대인관계로 힘들어하는 모습들도 보았습니다.

하지만 모두가 자신의 일을 사랑한다고 답했습니다. 일할 땐 힘들더라도 퇴근할 때에는 웃으며 동료들과 함께 돌아가는 모습을 보았습니다. 그런 사소하게 즐거운 순간들이 모여 삶을 만들어간다고 생각합니다. 제 자신에게 내린 답은, 주변에 어떤 색의 사람들이 있고 어떤 빛깔의 환경인지, 그곳에서 스스로의 만족을 찾아나가자는 것입니다.

브랜드 꼼데가르송와 꼼데가르송 한남 FSS 매장은 이러한 결정에 도달하기까지 저에게 많은 영감과 동기를 주었습니다. 매 시즌의 컬렉션과 의상들을 보고 있을 때, 말을 하지 않더라도 느껴지는 `절제된 미', 과하지 않으면서도 어디에서도 볼 수 없는 감각적인 테일러링을 선보이는 브랜드라고 생각합니다. 한남 FSS는 그러한 브랜드의 미학을 모두가 느낄 수 있도록 해주어, 제가 가장 사랑하는 장소 중 하나였습니다. 한남 FSS에서 스태프로 근무하며 제가 받았던 것과 같이 꼼데가르송이 갖는 아름다움를 사람들에게 전하고, 더 많은 사람들과 함께 공유할 수 있기를 원합니다.

\section*{꼼데가르송 한남 FSS 방문경험}
꼼데가르송 한남 FSS는 저의 대학교, 대학원 시절 많은 미학적 즐거움을 가져다주는 매장이었습니다. 한남동에 들를 때마다 시간이 나면 찾아오곤 했는데, 단순한 의류매장이 아닌 예술로써의 패션을 가장 가까운 곳에서 경험할 수 있게 해주는 곳이라고 생각합니다. 매력적인 층별 구성과 다른 곳에서는 볼 수 없는 꼼데가르송 각 라인의 오뜨쿠뛰르 의상도 볼 수 있어 매 방문 때마다 심적으로 즐겁습니다. 매 층마다 계시는 직원분들도 따라와주시며 부담되지 않게 친절히 안내해주시는 것도 좋았습니다.

1층의 PLAY와 CDGCDGCDG, Nike 제품들로 가볍게 시작하며, 4층에서부터 내려오면서 Comme des Garcons, Homme, Homme Plus, Homme Deux와 Black, Shirt, Junya Watanabe의 컬렉션까지 한 눈에 구경할 수 있습니다. 중간중간 내려오며 Jiyong Kim, Stussy, Brain Dead, Oakley과 같은 매력적인 다른 브랜드의 제품들도 DP되어있어 다양한 가격대의 다양한 제품들을 한 공간에서 만나볼 수 있었습니다.

가장 최근 방문은 1월 31일이었고, 인상적이었던 제품은: Junya Watanabe MAN-Oakley Factory Team 협업 슈즈, Homme Plus 23AW Lewis Leather 부츠, parfums의 향수 Concrete와 2였습니다.

\section*{외국어 구사 능력}
\begin{itemize}
    \item 영어
          \begin{itemize}
              \item 원어민 수준의 전문적 커뮤니케이션이 가능합니다.
              \item 스피킹 시험 OPI AH 등급(OPIc 상위호환 시험, 최고등급 바로 아래단계), 서울대학교 주관 종합영어능력평가 New TEPS 513/600 (TOEIC 환산 시 980/990에 해당)이 있습니다.
              \item 국제 학술지에 영문 논문 두 편 작성하여 출간하였고, 미국 국제 학회 영어 발표 경험 또한 보유하고 있어, 영어를 사용하는 의사소통은 막힘없이 가능합니다.
          \end{itemize}
    \item 일본어
          \begin{itemize}
              \item 일상적인 수준의 회화가 가능합니다.
              \item 일본 여행 시 이자카야나 기차, 페리 등에서 small talk를 하며 여러 친구를 만들었고, 가게나 공항 등에서도 거리낌없이 의사소통이 가능한 수준입니다.
              \item 자막 없이 일본 영화나 드라마를 시청하고 이해할 수 있습니다.
              \item 작문과 독해는 다소 부족하나, 면대면으로 의사소통 하는 데에 문제가 있던 경험은 없습니다 (번역기 사용X).
          \end{itemize}
\end{itemize}

\section*{스타일}
2019년부터 스케이트보드, 빈티지, 아카이브 패션을 거쳐 현재 하이엔드 디자이너 브랜드에 취향을 두고 있습니다.
현재 선호하는 브랜드는 Comme des Garcons 상위라인(Homme, Shirt, Black), Y/Project, Edward Cuming, Jean Paul Gaultier, GmbH, Simone Rocha, Bed JW Ford, TheSoloist, Ottolinger, Ann Demeulmeester, Kiko Kostadinov 등입니다.

\section*{기타 업무 수행능력}
\begin{itemize}
    \item 패션 트렌드에 민감하여 해외 디자이너 브랜드 직구, 배대지, FTA 관세협정 적용, 세컨핸즈(Grailed, Merucari, Vestiaire) 이용 경험이 많고, 의류 무역 전반 프로세스에도 친숙합니다.
    \item MS Word/Excel, Python 활용 다양한 문서행정 및 업무자동화 작업에 능숙합니다 (대학원 대표 행정조교 역임).
    \item 미팅, 컨퍼런스 등에서의 PPT 발표 경험이 많아 자료정리, 시각화, 청중 전달에 뛰어납니다  (국내외 학회발표 5회 및 연구미팅 발표경험 다수, 영어 발표 가능).
    \item 공학계열 석사학위자로, 빅데이터 처리 및 분석에 능숙합니다. 이를 활용하여 데이터 기반의 논리적 의사결정이 가능합니다 (인공지능 논문 제1저자 2편 게재 및 다수의 빅데이터/AI 관련 프로젝트 및 국책과제 수행 경험).
    \item 3년 간의 공학연구 수행으로 독립적이고 능동적인 업무 수행능력을 보유하고 있어, 업무 적응력과 학습 능력이 뛰어납니다.
\end{itemize}