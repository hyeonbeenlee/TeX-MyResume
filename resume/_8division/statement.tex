% 채용공고:
% 비주얼&SNS 컨텐츠 디렉터 - 1명
% CX 팀 - 1명
% 오프라인세일즈 - 3명
 
% 주식회사 뉴스탠다드는 리테일 편집매장 8DIVISION, 자사 브랜드 운영 및 해외의 유망한 라이징 브랜드들을 발굴하여 국내유통을 전개하는 패션 회사입니다. 이번 채용공고를 통해 유망한 인재분들과 함께 한걸음 더 나아가고자 합니다.
 
% 1. 업무
% <오프라인팀> - 3명
% - 8DIVISION에서 가장 고객에 가까운 장소에서 고객의 경험을 극대화로 이끌어 줄 수 있는 세일즈 파트
% - 고객응대 및 전반적인 매장 관리 (디스플레이등)
% - 모델로서도 활동 가능하신 분 우대 (피팅 및 SNS노출등)
% - 브랜드관계자 및 해외 고객들과의 소통이 가능하신 분 우대
 
% ㅤ
% 2. 공통채용조건
% - 성별무관 (남성지원자의 경우 군필자)
% - 패션과 기획등에 열정이 많으신 분, 성실하고 꼼꼼하신 분
% - 경력자 우대
% - 팀내에서 원활한 커뮤니케이션이 가능하신 분
% - 패션에 대한 관심과 이해도 필수
 
% ㅤ
% 3. 급여조건
% 추후협의
% ㅤ
% 4. 근무환경
% - 근무시간: 주 5일 근무 / 8시간 (연장근무 시 수당지급)
% - 장소: 서울시 중구 퇴계로 18길 31 8DIVISION
% ㅤ
% 5. 접수기간
% 채용시까지
% ㅤ
% 6. 접수방법
% 이메일접수 / recruit@8division.com
% ㅤ
% 7. 지원절차
% ①서류전형(1차 합격자에 개별통보) -> ② 대면 면접 -> ③ 최종합격
% ㅤ
% 8. 제출서류
% -이력서 (본인사진, SNS ID기재 필수)
% -자기소개서 
% -모든서류는 자유형식 입니다.
% ㅤ
% 8DIVISION 과 함께 미래를 가꿔나가고자 하는 인재분들의 많은 지원 바랍니다.

\section*{지원 동기}
8division 오프라인 세일즈 스태프 지원자 이현빈입니다. 대학원 공학석사학위 취득하였으나, 삶에서 패션문화를 더 가까이 두며 살아가고자 지원하게 되었습니다.

공학계열 학사 및 석사 과정을 수학하며, 많은 전문지식 습득과 프로젝트 수행 및 논문출판에 따른 성취감을 느꼈습니다. 그랬던 저를 지금의 결정에 저를 도달하게 한 것은 과도한 업무량도 잦은 야근과 출장이 아닙니다. 제가 사랑하는 문화를 함께 향유할 사람이 없었다는 점입니다.

공학에 몸담으며 가장 힘겨웠던 점은 그런 주변 환경이었습니다. 함께 직접적으로 패션을 주제로 이야기하지 않아도, 서로가 서로를 알아채고 이해할 수 있는 사람들이 곁에 있어주길 바랬습니다. 나름대로 노력해보았지만, 업무에 치여 그럴 기회나 시간이 그리 많지 않았고 점차 웃는 날들이 줄어가기만 했습니다.

일련의 시간과 생각들을 거쳐 졸업한 뒤로는 주변을 제가 좋아하는 것들로 채워나가기로 결정했습니다. 내가 좋아하는 것이니 재미있고 멋있기만 할 것이라는 헛된 기대나 환상을 품고 들어서지는 않았습니다. 주변에 패션업계나 세일즈 직무에 종사하는 친구들이 다들 각자 근무 스케줄이나 고객응대, 매장 내 대인관계로 힘들어하는 모습들도 보았습니다.

하지만 모두가 자신의 일을 사랑한다고 답했습니다. 일할 땐 힘들더라도 퇴근할 때에는 웃으며 동료들과 함께 돌아가는 모습을 보았습니다. 그런 사소하게 즐거운 순간들이 모여 삶을 만들어간다고 생각합니다. 제 자신에게 내린 답은, 주변에 어떤 색의 사람들이 있고 어떤 빛깔의 환경인지, 그곳에서 스스로의 만족을 찾아나가자는 것입니다.

8division은 이러한 결정에 도달하기까지 저에게 많은 영감과 동기를 주었습니다. AESyncTx, FFFPostalService, XLIM, Juntae Kim, Cost per Kilo와 같은 국내의 신진 디자이너 브랜드부터 Nepenthes 계열, North Works, TheSoloist, SasquatchFabrix 등의 일본 유명 브랜드, \_J.L-A.L\_, GmbH, Egon Lab, Namacheko, Eckhaus Latta의 유럽 신진 디자이너들과 Magliano, Raf Simons, Stefan Cooke, Bed J.W. Ford와 같은 전통 유럽 브랜드까지 한 곳에서 즐길 수 있는, 국내에서 몇 안되는 샵 중 하나입니다. 덕분에 저는 매 시즌 8division에서 전개하는 의류 컬렉션들을 즐기며 다양한 의류 스타일을 경험하고, 제가 사랑하는 아름다움을 추구하는데 도움받을 수 있었습니다. 저는 8division에서 매장 오프라인 세일즈 스태프로 근무하며 샵이 전개하는 다양한 디자이너들의 아름다움을 사람들에게 전하며, 더 많은 사람들과 함께 제가 받은 경험을 공유하기를 원합니다.

\section*{8division 방문 및 이용 경험}
고객으로서 8division을 이용하며 가장 인상깊었던 점들은 `최고의 쇼핑경험'이라는 키워드로 귀결됩니다. 이는 크게 세 가지로 정리될 수 있습니다.

첫 번째는 수 많은 브랜드가 입점해있음에도 거의 모든 의류에 대해 자체 모델 착용샷과 자세한 사이즈 실측이 기재되어 있었다는 점입니다. 고객 입장에서는 사이즈와 착용감을 모르기에 구매를 망설이게 되는 경우가 많은데, 8division은 이러한 점에서 상세하고 친절한 가이드를 통해 더 나은 쇼핑 경험을 제공하고 있었던 점이 매우 인상깊었습니다.

두 번째는 배송받을 때마다 어떤 스태프 분께서 저의 주문을 포장하였는지와 함께 감사인사 카드가 제공되었다는 점입니다. 이러한 사소한 정성과 친절이 최고의 서비스를 받고 있다는 점으로 작용하여, 더욱 8division 샵을 애용하게된 계기가 되었던 것 같습니다.

세 번째는 오프라인 매장의 편안한 분위기입니다. 보통 의류편집샵을 오프라인으로 방문하게 되면, 무겁고 경직된 분위기에서 쇼핑을 해왔습니다. 그러나 8division 명동 매장 방문 시 밝은 조명과 함께 아늑하면서도 메탈 계열로 마무리된 세련된 분위기에서, 스태프 분들의 과하지 않으면서도 친절한 안내 덕분에 쇼핑 경험의 만족도가 극대화되었습니다. 덕분에 혼자 방문하였음에도 편안한 마음으로 다양한 의류 컬렉션을 피팅해보고, 구매할 수 있었습니다.

이와 함께 풍성한 마일리지 적립과 세일 혜택까지 있어, 8division에서 쇼핑하는 것은 제게 크나큰 즐거움이었습니다. 자체 발간 저널 계정 또한 꾸준히 팔로우하고 있고, 가끔씩 올라오는 Team 8division 사진을 볼 때마다 `저 곳에 함께이고 싶다' 하는 생각을 했습니다. 이번 채용 기회를 통해 오프라인 세일즈 팀으로서 이러한 최고의 쇼핑 경험을 제공하는데에 기여하고, 8division과 함께 더 많은 사람들과 패션을 통한 즐거움을 향유하고 싶습니다.


\section*{외국어 구사 능력}
\begin{itemize}
    \item 영어
          \begin{itemize}
              \item 원어민 수준의 전문적 커뮤니케이션이 가능합니다.
              \item 스피킹 시험 OPI AH 등급(OPIc 상위호환 시험, 최고등급 바로 아래단계), 서울대학교 주관 종합영어능력평가 New TEPS 513/600 (TOEIC 환산 시 980/990에 해당)이 있습니다.
              \item 국제 학술지에 영문 논문 두 편 작성하여 출간하였고, 미국 국제 학회 영어 발표 경험 또한 보유하고 있어, 영어를 사용하는 의사소통은 막힘없이 가능합니다.
          \end{itemize}
    \item 일본어
          \begin{itemize}
              \item 일상적인 수준의 회화가 가능합니다.
              \item 일본 여행 시 이자카야나 기차, 페리 등에서 small talk를 하며 여러 친구를 만들었고, 가게나 공항 등에서도 거리낌없이 의사소통이 가능한 수준입니다.
              \item 자막 없이 일본 영화나 드라마를 시청하고 이해할 수 있습니다.
              \item 작문과 독해는 다소 부족하나, 면대면으로 의사소통 하는 데에 문제가 있던 경험은 없습니다 (번역기 사용X).
          \end{itemize}
\end{itemize}

\section*{스타일}
2019년부터 스케이트보드, 빈티지, 아카이브 패션을 거쳐 현재 하이엔드 디자이너 브랜드에 취향을 두고 있습니다.
현재 선호하는 브랜드는 Comme des Garcons 상위라인(Homme, Shirt, Black), Y/Project, Edward Cuming, Jean Paul Gaultier, GmbH, Simone Rocha, Bed JW Ford, TheSoloist, Ottolinger, Ann Demeulmeester, Kiko Kostadinov 등입니다.

\section*{기타 업무 수행능력}
\begin{itemize}
    \item 패션 트렌드에 민감하여 해외 디자이너 브랜드 직구, 배대지, FTA 관세협정 적용, 세컨핸즈(Grailed, Merucari, Vestiaire) 이용 경험이 많고, 의류 무역 전반 프로세스에도 친숙합니다.
    \item MS Word/Excel, Python 활용 다양한 문서행정 및 업무자동화 작업에 능숙합니다 (대학원 대표 행정조교 역임).
    \item 미팅, 컨퍼런스 등에서의 PPT 발표 경험이 많아 자료정리, 시각화, 청중 전달에 뛰어납니다  (국내외 학회발표 5회 및 연구미팅 발표경험 다수, 영어 발표 가능).
    \item 공학계열 석사학위자로, 빅데이터 처리 및 분석에 능숙합니다. 이를 활용하여 데이터 기반의 논리적 의사결정이 가능합니다 (인공지능 논문 제1저자 2편 게재 및 다수의 빅데이터/AI 관련 프로젝트 및 국책과제 수행 경험).
    \item 3년 간의 공학연구 수행으로 독립적이고 능동적인 업무 수행능력을 보유하고 있어, 업무 적응력과 학습 능력이 뛰어납니다.
\end{itemize}